%
% 建议用 pdflatex 编译
%
\documentclass{article}

%%%%%===== 页面设置 =====================================================
\usepackage[a4paper,top=2.54cm,bottom=2.54cm,left=3.17cm,right=3.17cm,%
            includehead,includefoot]{geometry}

%%%%%===== 常用宏包 =====================================================
\usepackage{amsmath,amssymb,amsfonts,amsthm}
\usepackage{graphicx}
\usepackage{subfigure}
\usepackage{float}
\usepackage{xcolor}
\usepackage[numbers,square,sort&compress]{natbib}
\usepackage{hyperref}
  \hypersetup{colorlinks,citecolor=blue,linkcolor=blue,breaklinks=true}
\usepackage{booktabs}
\usepackage{colortbl}
\usepackage{caption}
\usepackage{enumitem}
\usepackage{epstopdf}
\usepackage{algorithm}
\usepackage{algpseudocode}
\usepackage{array}
\usepackage{bbding}
\usepackage{fancyhdr} % 页眉
\usepackage{fancyvrb} % 摘录 Verbatim 环境
\usepackage{longtable}
\usepackage{listings}
\usepackage{rotating,rotfloat} % 提供 sidewayfigure 和 sidewaystable 环境横排图表
\usepackage{yhmath} % \wideparen 弧 \adots

%%%%%===== 页眉页脚 =====================================================
\usepackage{fancyhdr}
\pagestyle{fancy}
\fancyhf{}
\renewcommand{\headrulewidth}{0pt}
\renewcommand{\sectionmark}[1]{\markboth{\uppercase{#1}}{}}
\chead{\leftmark}
\cfoot{\thepage}

%%%%%===== 行间距 ======================================================
\renewcommand{\baselinestretch}{1.1}

%%%%%===== 数学公式 =====================================================
\allowdisplaybreaks

%%%%%===== 定义定理 =====================================================
\newtheorem{theorem}{Theorem}[section]
\newtheorem{lemma}{Lemma}[section]
\newtheorem{corollary}[theorem]{Corollary}
\newtheorem{proposition}[theorem]{Proposition}
\newtheorem{definition}{Definition}[section]
\newtheorem{example}{Example}
\newtheorem{remark}{Remark}

%%%%%===== 自定义命令 ====================================================
\newcommand{\R}{\mathbb{R}}
\newcommand{\dis}{\displaystyle}


\DeclareMathOperator{\diag}{diag}

\begin{document}

\title{\uppercase{Title of the paper Title of \\ the paper Title of the paper}}

\author{
  Yu Xingchun
}

\date{\today}

\maketitle

\begin{abstract}

\end{abstract}


\section{Riemann Problem}
The reactive Euler equations in one-dimensional space can be written as the Euler Equations plus a chemical reaction term, i.e.,
\begin{equation}
\partial_t \mathbf{U} + \partial_x \mathbf{F}(\mathbf{U}) = \mathbf{S}(\mathbf{U}),
\end{equation}
with
\begin{equation}
\boldsymbol{U} = \begin{bmatrix} \rho \\ \rho u \\ E \\ \rho \lambda \end{bmatrix}, \quad
\boldsymbol{F} = \begin{bmatrix} \rho u \\ \rho u^2 + p \\  (E+p)u \\ \rho u \lambda  \end{bmatrix}, \quad
\boldsymbol{S} = \begin{bmatrix} 0 \\ 0 \\ 0 \\ \rho \lambda K\end{bmatrix}.
\end{equation}
\section{Combustion Waves}
\subsection{Detonation Waves}
\subsection{Deflagration Waves}


\section{Equations for Pressure and Velocity}

\section{Numerical Solution for Pressure}

\section{The Complete Solution}
\input{chapter/part3.tex} 
\cite{luNumericalStudyRiemann1998}
\bibliographystyle{unsrt}
\bibliography{ReactiveFlow}

\end{document}
